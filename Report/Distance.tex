\section{Distance}
\label{sec:Distance}

In order to measure the similarity of different samples, the Dynamic Time Warping (DTW) algorithm was used. DTW compares sequences of feature vectors to one another even if the sequences are of different lengths. This is important since the samples in our sample base have different lengths. DTW stretches and compresses samples feature vector sequences for maximum similarity, this finding similarities in samples that are elongated versions of one another or that contain each other.

DTW first compares every feature vector to every other feature vector by calculating the euclidean distance between those vectors. This creates a cost matrix $C$ of distances between every block of the first sample to every block to the second sample in the feature space. This matrix is of dimension $N \times M$, where $N$ and $M$ are the lengths of the feature sequences of both samples, respectively.

Now, DTW searches for the fastest path through $C$. In order to not having to evaluate every possible path, DTW only calculated the cheapest path from every positive time step from $C_{n-1,m}, C_{n,m-1}, C_{n-1,m-1}$ to $C_{n,m}$. The final distance between the samples is then calculated adding all the steps $C_{n.m}$ on the cheapest paths from $C_{0,0}$ to $C_{N,M}$. Additionally, every step is multiplied by $\frac{1}{N}$ or $\frac{1}{M}$ or $\sqrt{\frac{1}{N}^2 + \frac{1}{M}^2}$ to correct for the stepping distance.

This algorithm was implemented in Python, but found too slow for practical comparisons of big sample sets. Thus, we implemented it in C and called that version from Python, which provided two orders of magnitude of speedup.
