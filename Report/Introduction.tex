\section{Introduction}
\label{sec:Introduction}
Electronic Dance Music (EDM) is usually built from electronic sound sources such as synthesizers or drum machines. In the last decade, more and more software products such as Digital Audio Workstations (DAW) and software instruments have been used as well. One of the most important sound generating procedures is sampling. In this technique small audio snippets that are played back to generate a sound.\\
These samples could be base elements such as single snare hits, or whole rhythmic loops, or even complex drum arrangements. In this project, we assume that electronic dance music is comprised only of samples. Thus, complex musical productions should be representable as superpositions of many short samples. We liken these samples to words in speech in that a song is made of samples much like a sentence is made of words.\\
During the production of electronic dance music, it is often interesting to find similar samples to a known one. The process of choosing these samples can be very complex. We created an algorithm to aid in this selection process. The ideal case would be to find similar samples to all samples used in a track. \\
A conceptual framework for this would be a three step system: First, an existing collection of samples is divided into different classes. Second, short parts of a musical track are classified according to these classes. For typical tracks, there will be no clear classification, but more likely a superposition of different probabilities for each short part. Lastly, these classifications can be used to match the musical track back to the samples, thus re-synthesizing the music. In order to do this, the classes do not need to correspond to musically relevant classes such as bass drums or string instruments.\\
In this project, we are doing some basic research on this topic. We create new classes from a sample database using a clustering algorithm, and then classify test samples from these classes. Also, we classified our sample database by hand, and then classify test samples to these classes.
