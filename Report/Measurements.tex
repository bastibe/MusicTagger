\section{Measurements and Results}
\label{sec:Measurements}
A test sample base was created with ten tracks for each class. All of the samples where not included in the training sample base. 

\subsection{$k$-Means}
The number of tags within each cluster created by the k-Means algorithm where counted. The results are shown in figure~\ref{}.

It can be seen that the tags defined by our sample base do not represent the clusters which were automatically defined. This was to be expected. Nevertheless it seems that there are some consistent mappings in the structuring of the tags. This needs to be verified by further reseach. 


\subsection{$k$-Nearest-Neighbors}
All of the samples of the test bank were classified using the implemented $k$-NN algorithm with $k=10$. The results are shown in figure~\ref{}. The Tags of the test bank are plottet against the tags of the classification. It can be seen that many of the samples are classified right. The results overcame the expectations. The variations that can be seen in the classification correlate with variances in the human classification. This means the classes which are bad classified are also widely spread in their human perception. 