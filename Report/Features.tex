\section{Features}
\label{sec:Features}
Each sample is split into windowed and overlapping blocks. Different features are calculated on each block. The blocks have a length of 20~ms, 50~\% overlap and are Hann windowed.\\
Features used are the RMS, peak, crest factor, spectral centroid, logarithmic spectral centroid, spectral variance, spectral skewness, spectral flatness, spectral brightness and mean spectral absolute slope. They are described in detail as follows.\\
For every feature, the block is given as $x$, $x[k]$ is a frame within the block, $K$ is the number of frames within one block. The FFT of the block is $X$, $X[n]$ is an FFT bin, $2 N$ is the FFT-Length. The features are calculated as follows:
\begin{description}
    \item[RMS:]
        The RMS is a measure for the energy of the block.
        \[
            \mathrm{RMS} = \sqrt{\frac{1}{K}\sum_{k=0}^{K-1}{x[k]^2}}
        \]
    \item[Peak:]
        The peak is simply the maximum of all absolute values within the block.
        \[
            \mathrm{PK} = \max{|x[k]|}
        \]
    \item[Crest Factor:]
        The ratio of the peak to the RMS value is the crest factor.
        \[
            \mathrm{CF} = \frac{\mathrm{PK}}{\mathrm{RMS}}
        \]
    \item[Spectral Centroid:]
        The spectral centroid is the normalized frequency weighted mean of the absolute spectrum. The frequency is normalized to the sampling frequency.
        \[
            \mathrm{SC} = \sum_{n=0}^{N}{|X[n]| \cdot \frac{n}{2N}}
        \]

    \item[Logarithmic Spectral Centroid:]
        To account for human frequency perception, the spectral centroid is calculated relative to a logarithmic frequency axis as well.
        \[
            \mathrm{LSC} = \sum_{n=0}^{N}{|X[n]| \cdot \log{\frac{n}{2 N}+1}}
        \]
    \item[Spectral Variance:]
        The spectral variance gives a value of the change of the absolute spectrum over all frequencies.
        \[
            \mathrm{SV} = \sum_{n=0}^{\frac{K}{2}}{(|X[n]|-\overline{|X[n]|})^2}
        \]
        with the spectral mean
        \[
            \overline{|X[n]|} = \frac{1}{N+1}\sum_{n=0}^{N}{|X[n]|}.
        \]
    \item[Spectral Skewness:]
    		The skewness of the absolute spectrum shows how much its distribution leans towards high or low frequencies.
        \[
            \mathrm{SS} = \sum_{n=0}^{\frac{K}{2}}{(|X[n]|-\overline{|X[n]|})^3}.
        \]
    \item[Spectral Flatness:]
        Given the spectral mean $\overline{|X[n]|}$ and the geometric spectral mean
        \[
            \overline{|X[n]|}_\mathrm{geom} = \sqrt[N+1]{\prod_{n=0}^{N}{|X[n]|}},
        \]
        the spectral flatness is calculated as the ratio of them.
        \[
        		\frac{\overline{|X[n]|}_\mathrm{geom}}{\overline{|X[n]|}}
        \]
    \item[Spectral Brightness:]
        The spectral brightness is the ratio of high to the low frequency energy. To get the energy of low and high frequency, the weighted sum of the absolute spectrum is taken. The weighting of the high frequency bins is given as
        \begin{eqnarray*}
            w_\mathrm{high} &=& 0.5 - \frac{\cos(\omega)}{2}\\
            w_\mathrm{low} &=& \frac{\cos(\omega)}{2} + 0.5
        \end{eqnarray*}
        with $\omega$ as $N$ logarithmic ordered values in the range $[0;~\pi]$. The point of intersection between the low and high frequency weighting was set to approximately 2~kHz.
        \[
            \mathrm{SB} = \frac{\sum_{n=0}^{\frac{K}{2}}{|X[n]|\cdot w_\mathrm{high}}}{\sum_{n=0}^{\frac{K}{2}}{|X[n]|\cdot w_\mathrm{low} }}
        \]
    \item[Mean Absolute Spectral Slope:]
        The mean of the absolute spectral slope gives the amount of change in the spectrum over the frequency. It is calculated by
        \[
            \mathrm{MASS} = \frac{1}{N}\sum_{n=1}^{N}{||X[n]|-|X[n-1]||}
        \]
\end{description}
The features of all blocks of all samples are saved within a Pandas DataFrame\footnote{http://pandas.pydata.org/}. This makes them easily searchable and saveable.